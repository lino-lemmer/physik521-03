% Für Seitenformatierung

\documentclass[DIV=15]{scrartcl}

% Zeilenumbrüche

\parindent 0pt
\parskip 6pt

% Für deutsche Buchstaben und Synthax

\usepackage[ngerman]{babel}

% Für Auflistung mit speziellen Aufzählungszeichen

\usepackage{paralist}

% zB für \del, \dif und andere Mathebefehle

\usepackage{amsmath}
\usepackage{commath}
\usepackage{amssymb}

% für nicht kursive griechische Buchstaben

\usepackage{txfonts}

% Für \SIunit[]{} und \num in deutschem Stil

\usepackage[output-decimal-marker={,}]{siunitx}
\usepackage[utf8]{inputenc}

% Für \sfrac{}{}, also inline-frac

\usepackage{xfrac}

% Für Einbinden von pdf-Grafiken

\usepackage{graphicx}

% Umfließen von Bildern

\usepackage{floatflt}

% Für Links nach außen und innerhalb des Dokumentes

\usepackage{hyperref}

% Für weitere Farben

\usepackage{color}

% Für Streichen von z.B. $\rightarrow$

\usepackage{centernot}

% Für Befehl \cancel{}

\usepackage{cancel}

% Für Layout von Links

\hypersetup{
	citecolor=black,
	colorlinks=true,
	linkcolor=black,
	urlcolor=blue,
}

% Verschiedene Mathematik-Hilfen

\newcommand \e[1]{\cdot10^{#1}}
\newcommand\p{\partial}

\newcommand\half{\frac 12}
\newcommand\shalf{\sfrac12}

\newcommand\skp[2]{\left\langle#1,#2\right\rangle}
\newcommand\mw[1]{\left\langle#1\right\rangle}
\renewcommand \exp[1]{\mathrm e^{#1}}

% Nabla und Kombinationen von Nabla

\renewcommand\div[1]{\skp{\nabla}{#1}}
\newcommand\rot{\nabla\times}
\newcommand\grad[1]{\nabla#1}
\newcommand\laplace{\triangle}
\newcommand\dalambert{\mathop{{}\Box}\nolimits}

%Für komplexe Zahlen

\renewcommand \i{\mathrm i}
\renewcommand{\Im}{\mathop{{}\mathrm{Im}}\nolimits}
\renewcommand{\Re}{\mathop{{}\mathrm{Re}}\nolimits}

%Für Bra-Ket-Notation

\newcommand\bra[1]{\left\langle#1\right|}
\newcommand\ket[1]{\left|#1\right\rangle}
\newcommand\braket[2]{\left\langle#1\left.\vphantom{#1 #2}\right|#2\right\rangle}
\newcommand\braopket[3]{\left\langle#1\left.\vphantom{#1 #2 #3}\right|#2\left.\vphantom{#1 #2 #3}\right|#3\right\rangle}


\setcounter{section}{0}
\renewcommand\thesection{H\,3.\arabic{section}}
\renewcommand\thesubsection{\thesection.\alph{subsection}}

\title{physik521: Übungsblatt 03}
\author{%
    Lino Lemmer \\ \small{\texttt{s6lilemm@uni-bonn.de}}
    \and
    Martin Ueding \\ \small{\texttt{mu@martin-ueding.de}}
    \and
    Paul Manz \\ \small{\texttt{p.m@uni-bonn.de}}
}

\begin{document}
\maketitle

\section{Thermodynamische Beziehungen in einem magnetischen System}

\newcommand\tdpd[3]{\del{\dpd{#1}{#2}}_{#3}}

\subsection{Erste Herleitung}

Es ist zu zeigen, dass gilt:
\[
    \frac{c_B}{c_M} = \frac{\chi_T}{\chi_S}
\]

Im Skript ab Seite 42 ist diese Aufgabe für zwei andere Größen
vorgerechnet. Ich habe diese Lösung versucht nachzuvollziehen und hier
mit mehr Erklärungen für die magnetischen Größen wiederzugeben.

Hier sind alle Umformungen, die Erklärungen folgen:
\[
    \frac{c_B}{c_M}
    \overset{\ref{1-1}}=
    \frac{T \tdpd S T B}{T \tdpd S T M}
    \overset{\ref{1-2}}=
    \frac{ \frac{\tdpd BTS}{\tdpd BST} }{ \frac{\tdpd MTS}{\tdpd MST} }
    =
    \frac{\tdpd BTS}{\tdpd MTS} \frac{\tdpd MST}{\tdpd BST}
    \overset{\ref{1-3}}=
    \frac{\tdpd BTS \tdpd TMS}{\tdpd BST \tdpd SMT}
    \overset{\ref{1-4}}=
    \frac{\tdpd MBS}{\tdpd BMT}
    \overset{\ref{1-1}}=
    \frac{\chi_T}{\chi_S}
\]

\begin{enumerate}
    \item
        \label{1-1}
        Dies gilt nach Aufgabenstellung.

    \item
        \label{1-2}
        Wir sehen $B$ als Funktion von $S$ und $T$ an und erhalten somit
        \[
            0 = \dif B = \tdpd BST \dif S + \tdpd BTS \dif T,
        \]
        welches wir, unter Benutzung des Kehrwertes der Ableitungen und der
        Kettenregel, zu
        \[
            \frac{\dif S}{\dif T} = - \frac{\tdpd BTS}{\tdpd BST} = \tdpd STB
        \]
        umstellen können.

        Analog können wir auch herleiten, dass gilt:
        \[
            \tdpd STM = - \frac{\tdpd MTS}{\tdpd MST}.
        \]

    \item
        \label{1-3}
        Hier nutzen wir aus, dass der Kehrwert einer solchen partiellen
        Ableitung die Differenzierung umkehrt.

    \item
        \label{1-4}
        An dieser Stelle benutzen wir die Kettenregel
        \[
            \tdpd \alpha\delta\gamma = \tdpd \alpha\beta\gamma \tdpd \beta\delta\gamma
        \]
        im Zähler und Nenner jeweils um die Zwischenvariable $S$ zu
        eliminieren.
\end{enumerate}

\subsection{Zweite Herleitung}

Diese Aufgabe wird so komplett ab Seite 42 im Skript vorgerechnet. Hier werden
wir die Rechungen aus dem Skript repoduzieren und versuchen, sie ausführlicher
als im Skript zu erläutern, damit es auch etwas zur Übung beiträgt.

Nach Skript folgt können $c_p$ und $c_v$ durch partielle Ableitungen von $S$
ausgedrückt werden:
\begin{align*}
    c_p - c_v
    &= T \del{\tdpd STp - \tdpd STV} \\
    \intertext{%
        Wir beginnen damit, die erste Ableitung anders auszudrücken. Dazu
        interpretieren wir $S$ nicht als $S(T, V, p)$, sondern als $S(T, V(p,
        T))$. Wenn wir jetzt nach $T$ ableiten, erhalten wir zwei Terme durch
        die mehrdimensionale Kettenregel. Jedoch würden wir in dieser Stelle
        eigentlich erwarten, dass die partielle Ableitung von $S$ nach $T$
        nicht Term mit $S$ nach $V$ enthält, da nur partiell und nicht total
        nach $T$ abgeleitet wird. Jedoch wurde dies schon an einigen Stellen so
        gemacht und scheint auch richtig zu sein. Wir wissen nur noch nicht,
        warum dies so ist.
    }
    &= T \del{\tdpd STV + \tdpd SVT \tdpd VTp - \tdpd STV} \\
    &= T \tdpd SVT \tdpd VTp \\
    \intertext{%
        Der letzte Faktor beschreibt gerade die isobare Expansion. Diese wird
        durch den Expansionskoeffizienten $\alpha_p$ ausgedrückt, der wie folgt
        definiert ist:
        \[
            \alpha_p := \frac 1V \tdpd VTp.
        \]
        Somit erhalten wir:
    }
    &= T \tdpd SVT V \alpha_p \\
    \intertext{%
        Nun müssen wir die freie Energie $F$ benutzen. Damit können wir die
        Ableitung $S$ nach $V$ ausdrücken. Es gilt:
        \[
            S = - \tdpd FTV.
        \]
        Dies leiten wir nun nach dem Volumen $V$ ab, wobei wir die Temperatur
        $T$ festhalten:
        \[
            \tdpd SVT = - \tdpd {}VT \tdpd FTV.
        \]
        Damit erhalten wir:
    }
    &= - T \tdpd {}VT \tdpd FTV V \alpha_p \\
    \intertext{%
        Da $F$ ein exaktes Differential hat, kommutieren die partiellen
        Ableitungen hier.
    }
    &= - T \tdpd {}TV \tdpd FVT V \alpha_p \\
    \intertext{%
        Die partielle Ableitung der freien Energie nach dem Volumen ist gerade
        der negative Druck.
    }
    &= T \tdpd pTV V \alpha_p \\
    \intertext{%
        Mit der Annahme, dass $\dif V = 0$ ist, können wir noch das
        Differential von $V(p, T)$ schreiben als:
        \[
            \tdpd VpT \dif p + \tdpd VTp \dif T = 0.
        \]
        Mit dem gleichen Trick aus der vorherigen Aufgabe kann damit nach $\dif
        p / \dif T$ umgestellt werden. Da wir $\dif V = 0$ angesetzt hatten,
        müssen wir jetzt $V$ festhalten. Warum wir dies jetzt als partielle
        Ableitung schreiben können, ist uns noch nicht ganz klar. Jedenfalls
        erhalten wir dann:
        \[
            \tdpd pTV = - \frac{\tdpd VTp}{\tdpd VpT}.
        \]
        Dies setzen wir ein.
    }
    &= - T \frac{\tdpd VTp}{\tdpd VpT} V \alpha_p \\
    \intertext{%
        Im Zähler steht wieder der Ausdehungskoeffizient $\alpha_p$. Im Nenner
        steht die isotherme Kompressibilität, die angibt, wie sich das Volumen
        mit dem Druck verändert. Somit erhalten wir:
    }
    &= - T \frac{V \alpha_p}{- V \kappa_T} V \alpha_p \\
    &= TV \frac{\alpha_p^2}{\kappa_T} \\
\end{align*}

Damit ist die Relation gezeigt.

\subsection{Dritte Herleitung}

Nun das gleiche für $c_B$ und $c_M$. Wir nehmen die Herleitung aus der
vorherigen Aufgabe und ersetzen die Größen, lassen aber die Erklärungen jetzt
weg. $p \mapsto B$, $V \mapsto M$.

\begin{align*}
    c_B - c_M
    &= T \del{\tdpd STB - \tdpd STM} \\
    &= T \del{\tdpd STM + \tdpd SMT \tdpd MTB - \tdpd STM} \\
    &= T \tdpd SMT \tdpd MTB \\
    &= T \tdpd SMT \alpha_B \\
    &= - T \tdpd {}MT \tdpd FTM  \alpha_p \\
    &= - T \tdpd {}TM \tdpd FMT \alpha_p \\
    &= T \tdpd BTM \alpha_B \\
    &= - T \frac{\tdpd MTB}{\tdpd MBT} \alpha_B \\
    &= - T \frac{\alpha_B}{- \kappa_T} \alpha_B \\
    &= T \frac{\alpha_B^2}{\kappa_T} \\
\end{align*}

\section{Joule-Thomson-Effekt}
\subsection{Enthalpie und Irreversiblität}
Stelle zunächst fest, dass Teilchenzahl und Wärmeenergie unverändert bleiben, da wir einen adiabatischen Prozess in einem abgeschlossenen System betrachten.
\begin{align*}
    \dif N &= 0 \\
    \delta Q &= 0 \\
    \intertext{%
        Der erste Hauptsatz führt dann auf:
    }
    \Delta U &= U_2 - U_1 \\
             &= \Delta W \\
             &= -\int_{V_1}^{0} p_1 \dif V  + \int_{0}^{V_2} p_2 \dif V \\
             &= p_1 V_1 - p_2 V_2
    \intertext{%
        Für die Enthalpie gilt also:
    }
    H &= U+pV \\
    \implies \Delta H &= U_1 + p_1 V_1 - (U_2 + p_2 V_2) =0
    \intertext{%
        Aus der Definition der Enthalpie folgt:
    }
    \dif H &= T \dif S + V \dif P + \mu \dif N = 0 \\
    \iff T \dif S &= V \dif P > 0 = \delta Q
\end{align*}
Der Prozess läuft also irreversibel ab.

\subsection{Joule-Thomson-Koeffizient}
Der Joule-Thomson-Koeffizient ist definiert als:
\begin{align*}
    \delta &= \del {\pd{T}{p}}_H
    \intertext{%
        Wir wissen:
    }
    \dif H &= T \dif S + V \dif p + \mu \dif N \\
    S\del{T,p,N} \implies \dif S &= \del {\pd ST}_{p,N} \dif T +
    \del{\pd Sp}_{T,N} \dif p
    \intertext{%
        Verwende $\dif N = 0$ und setze ein:
    }
    \dif H &= T \del {\pd ST}_{p,N} \dif T + T \del{\pd Sp}_{T,N} \dif p +
    V \dif p \\
    \dif H &= C_p \dif T + T \del{\pd Sp}_{T,N} \dif p + V \dif p \\
    0 &= C_p \del {\pd{T}{p}}_H + T \del{\pd Sp}_{T,N}+ V \\
    0 &= C_p \delta + T \del{\pd Sp}_{T,N}+ V \\
    \delta &= \frac{1}{C_p} \del { -T \del{\pd Sp}_{T,N}- V}
    \intertext{%
        Wir müssen noch eine Maxwell-Relation aus der Gibbs freien Energie herleiten:
    }
    \dif G &= -S \dif T + V \dif p + \mu \dif N \\
    \del {\pd GT}_{P,N} &= -S \\
    \del {\pd Gp}_{T,N} &= V \\
    \implies -\del{\pd Sp}_{T,N} &= \del{\pd VT}_{P,N}
    \intertext{%
    Eingesetzt führt das auf das Endergebnis}
    \delta &= \frac{1}{C_p} \del { T \del{\pd VT}_{P,N}- V}
    \intertext{%
        Für das ideale Gas gilt:
    }
    V &= \frac{Nk_B T}{p} \\
    \implies T\del{\pd VT}_{N,P}-V &= \frac{Nk_B T}{p} -\frac{Nk_B T}{p} = 0 \\
    \implies \delta &= 0
\end{align*}


\subsection{Joule Kreisprozess}


\section{Elastischer Draht}
\subsection{}
Der Draht habe eine konstante Querschnittsfläche $A$. Dann gilt für die Länge des Drahtes $L=\frac VA$ und für die Kraft $K=Ap$. Für das Differential der inneren Energie $U\del {S,L}$ gilt dann:
\begin{align*}
    \dif U &= \del{\pd US}_L \dif S + \del{\pd UL}_S \dif L \\
           &= T \dif S + \del{\pd VL}_S \del{\pd UV}_S \dif L \\
           &= T \dif S - \frac{p}{A} \dif L \\
           &= T \dif S - K \dif L
    \intertext{%
        Wir wollen nun ein thermodynamisches Potential $M\del{L,T}$ einführen. Führe Legendre-Trafo durch:
    }
    M\del{L,T} &\coloneqq U\del{S,L}- S \del{\pd US}_L \\
                &= U-TS \\
    \dif M &= T \dif S - K \dif L - T \dif S - S \dif T \\
           &=-F \dif L - S \dif T
    \intertext{%
        Dieses Potential können wir zur Herleitung folgender Maxwell-Relation benutzen:
    }
    \del{\pd ML}_T &= -K \\
    \del{\pd MT}_L &= -S \\
    \implies \del{\pd KT}_L &= \del{\pd SL}_T
\end{align*}


\subsection{}
Aufgrund der Homogenität des Drahtes und der Tatsache, dass $S$, $L$ und $U$ extensive Größen sind folgt:
\begin{align*}
    S \del{\lambda U,\lambda L} &= \lambda S\del{U,L} \\
    \intertext{%
        Differenziation nach $\lambda$ für $\lambda \to 1$ ergibt dann:
    }
    \dod{}{\lambda} \del{\lambda S}= S &=\del{\pd SU}_L U + \del{\pd SL}_U L \\
                                                 &= \frac{U}{T}+ FL
\end{align*}

\end{document}
